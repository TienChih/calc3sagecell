\documentclass{ximera}
      
\title{Local Extrema}
      
\begin{document}
      
\begin{abstract}
      
In this activity, we use the Frobenius method of identifying local extrema.
      
\end{abstract}
      
\maketitle
      

Enter the function $f(x,y)$.  The SageCell then identifies the first and second derivatives, identifies the real critical points, and uses the Frobenius method to categorize them as local maxima, minima or saddle points.


\begin{sageCell}
var('x y'); 

f(x,y)= x^3-2*x*y

fx=diff(f,x)
fy=diff(f,y)

fxx=diff(fx,x)
fxy=diff(fx,y)
fyy=diff(fy,y)

f
print('The derivative fx is:'+`fx`)
print('The derivative fy is:'+`fy`)
print('The second derivative fxx is:'+`fxx`)
print('The second derivative fxy is:'+`fxy`)
print('The second derivative fyy is:'+`fyy`)


cp=solve([fx==0, fy==0], x, y); cp

critpoint= []

for i in range(len(cp)):
    if cp[i][0].right() in RR:
        critpoint.append((cp[i][0].right(), cp[i][1].right()))
        
print('The critical points are '+`critpoint`)



for j in range(len(critpoint)):
    if ((fxx*fyy-fxy^2)(critpoint[j][0], critpoint[j][1])>0) and fxx(critpoint[j][0], critpoint[j][1])>0:
        critpoint[j]
        print('is a min')
    elif ((fxx*fyy-fxy^2)(critpoint[j][0], critpoint[j][1])>0) and fxx(critpoint[j][0], critpoint[j][1])<0:
        critpoint[j]
        print('is a max')
    elif (fxx*fyy-fxy^2)(critpoint[j][0], critpoint[j][1])<0:
        critpoint[j]
        print('is a saddle point')
    else:
        critpoint[j]
        print('is something')
\end{sageCell}





 
 
 
 
      






\end{document}
